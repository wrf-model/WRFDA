\startmanpage
\mantitle{MPI{\tt \char`\_}Comm{\tt \char`\_}dup}{tex}{10/14/2006}
\manname{MPI{\tt \char`\_}Comm{\tt \char`\_}dup}
--- Duplicates an existing communicator with all its cached information 
\subhead{Synopsis}
\startvb\begin{verbatim}
int MPI_Comm_dup(MPI_Comm comm, MPI_Comm *newcomm)

\end{verbatim}
\endvb

\subhead{Input Parameter}
\startarg{comm }{Communicator to be duplicated (handle) 
}
\par
\subhead{Output Parameter}
\startarg{newcomm }{A new communicator over the same group as {\tt comm} but with a new
context. See notes.  (handle) 
}
\par
\subhead{Notes}
This routine is used to create a new communicator that has a new
communication context but contains the same group of processes as
the input communicator.  Since all MPI communication is performed
within a communicator (specifies as the group of processes {\em plus
}the context), this routine provides an effective way to create a
private communicator for use by a software module or library.  In
particular, no library routine should use {\tt MPI{\tt \char`\_}COMM{\tt \char`\_}WORLD} as the
communicator; instead, a duplicate of a user-specified communicator
should always be used.  For more information, see Using MPI, 2nd
edition.
\par
Because this routine essentially produces a copy of a communicator,
it also copies any attributes that have been defined on the input
communicator, using the attribute copy function specified by the
{\tt copy{\tt \char`\_}function} argument to {\tt MPI{\tt \char`\_}Keyval{\tt \char`\_}create}.  This is
particularly useful for (a) attributes that describe some property
of the group associated with the communicator, such as its
interconnection topology and (b) communicators that are given back
to the user; the attibutes in this case can track subsequent
{\tt MPI{\tt \char`\_}Comm{\tt \char`\_}dup} operations on this communicator.
\par
\subhead{Thread and Interrupt Safety}
\par
This routine is thread-safe.  This means that this routine may be
safely used by multiple threads without the need for any user-provided
thread locks.  However, the routine is not interrupt safe.  Typically,
this is due to the use of memory allocation routines such as {\tt malloc
}or other non-MPICH runtime routines that are themselves not interrupt-safe.
\par
\subhead{Notes for Fortran}
All MPI routines in Fortran (except for {\tt MPI{\tt \char`\_}WTIME} and {\tt MPI{\tt \char`\_}WTICK}) have
an additional argument {\tt ierr} at the end of the argument list.  {\tt ierr
}is an integer and has the same meaning as the return value of the routine
in C.  In Fortran, MPI routines are subroutines, and are invoked with the
{\tt call} statement.
\par
All MPI objects (e.g., {\tt MPI{\tt \char`\_}Datatype}, {\tt MPI{\tt \char`\_}Comm}) are of type {\tt INTEGER
}in Fortran.
\par
\subhead{Errors}
\par
All MPI routines (except {\tt MPI{\tt \char`\_}Wtime} and {\tt MPI{\tt \char`\_}Wtick}) return an error value;
C routines as the value of the function and Fortran routines in the last
argument.  Before the value is returned, the current MPI error handler is
called.  By default, this error handler aborts the MPI job.  The error handler
may be changed with {\tt MPI{\tt \char`\_}Comm{\tt \char`\_}set{\tt \char`\_}errhandler} (for communicators),
{\tt MPI{\tt \char`\_}File{\tt \char`\_}set{\tt \char`\_}errhandler} (for files), and {\tt MPI{\tt \char`\_}Win{\tt \char`\_}set{\tt \char`\_}errhandler} (for
RMA windows).  The MPI-1 routine {\tt MPI{\tt \char`\_}Errhandler{\tt \char`\_}set} may be used but
its use is deprecated.  The predefined error handler
{\tt MPI{\tt \char`\_}ERRORS{\tt \char`\_}RETURN} may be used to cause error values to be returned.
Note that MPI does {\em not} guarentee that an MPI program can continue past
an error; however, MPI implementations will attempt to continue whenever
possible.
\par
\startarg{MPI{\tt \char`\_}SUCCESS }{No error; MPI routine completed successfully.
}
\startarg{MPI{\tt \char`\_}ERR{\tt \char`\_}COMM }{Invalid communicator.  A common error is to use a null
communicator in a call (not even allowed in {\tt MPI{\tt \char`\_}Comm{\tt \char`\_}rank}).
}
\par
\subhead{See Also}
 MPI{\tt \char`\_}Comm{\tt \char`\_}free, MPI{\tt \char`\_}Keyval{\tt \char`\_}create, MPI{\tt \char`\_}Attr{\tt \char`\_}put, MPI{\tt \char`\_}Attr{\tt \char`\_}delete,
\nextline
MPI{\tt \char`\_}Comm{\tt \char`\_}create{\tt \char`\_}keyval, MPI{\tt \char`\_}Comm{\tt \char`\_}set{\tt \char`\_}attr, MPI{\tt \char`\_}Comm{\tt \char`\_}delete{\tt \char`\_}attr
\location{comm{\tt \char`\_}dup.c}
\endmanpage
