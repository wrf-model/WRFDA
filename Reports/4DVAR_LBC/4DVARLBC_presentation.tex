% WRF4DVar LBC
% March 2010 
% Autor: Xin Zhang 
% MMM, National Center for Atmospheric Research
% www.mmm.ucar.edu
% email: xinzhang@ucar.edu

\documentclass{beamer}
%\usepackage{pgfpages}
%\pgfpagesuselayout{4 on 1}[a4paper, landscape,border shrink=10mm]
\usepackage{amsmath,amssymb}
\usepackage{times}
\setbeamercovered{dynamic}
\setbeamertemplate{navigation symbols}{}
\usepackage[english]{babel}
\usetheme{Warsaw}
\usepackage{graphics}
\usepackage{hyperref}
\beamersetuncovermixins{\opaqueness<1>{25}}{\opaqueness<2->{15}}
\begin{document}

\title[LBC control in WRF 4D-Var]{Development of Lateral Boundary Control \\in WRF 4D-Var System}

\author[Xin Zhang et al.]{Xin Zhang\inst{1} \and Hans Huang\inst{1} \and Nils Gustafsson\inst{2}}
\institute[NCAR]{\inst{1} National Center for Atmospheric Research, Boulder, CO USA \and \inst{2} Swedish Meteorological and Hydrological Institute, Norrkoping, Sweden}
\date{The 4th East Asia WRF Workshop}
\logo{\includegraphics[scale=0.13]{ncar_logo}}


\frame{\titlepage}

\begin{frame}
\frametitle{Outline} 
\tableofcontents
\end{frame}

\section{WRFDA in WRF Modeling System}

\begin{frame}
\frametitle{WRFDA in WRF Modeling System}
\end{frame}

\subsection{Prepare the BE}
\begin{frame}
\frametitle{Prepare the BE}
\begin{equation*}
J(\bf{x})=\frac{1}{2}(\bf{x}-\bf{x}^b)^T{\color{red}\bf{B}^{-1}}(\bf{x}-\bf{x}^b)+\frac{1}{2}(\bf{y}-\textsl{H}(\bf{x}))^T\bf{R}^{-1}(\bf{y}-\textsl{H}(\bf{x}))
\end{equation*}
\begin{itemize}
\item Nominally, {\color{red}$\mathbf{B}$} is the background error covariance \pause
\item For initial testing, default background error statistics may
be used \pause 
\begin{itemize}
\item be.dat file (CV option 5) from test case tar file can only be used
with the domain from online tutorial \pause
\item be.dat.cv3 (CV option 3) from source code tar file can be used for
general test domains 
\end{itemize} \pause
\item Ultimately, {\color{red}$\mathbf{B}$} should be specific to the particular
model domain (and season)
\end{itemize}
\end{frame}

\subsection{Prepare the Background}
\begin{frame}
\frametitle{Prepare the Background}
\begin{equation*}
J(\bf{x})=\frac{1}{2}(\bf{x}-{\color{red}\bf{x}^b})^T{\color{blue}\bf{B}^{-1}}(\bf{x}-{\color{red}\bf{x}^b})+\frac{1}{2}(\bf{y}-\textsl{H}(\bf{x}))^T\bf{R}^{-1}(\bf{y}-\textsl{H}(\bf{x}))
\end{equation*}
\begin{itemize}
\item In “cold-start” mode: accomplished by running the
WPS and real programs \pause
\begin{itemize}
\item The background is essentially the wrfinput\_d01 file
\end{itemize} \pause
\item In “cycling” mode: the output of the WRF model \pause
\begin{itemize}
\item WRF can output wrfinput-formatted files used for cycling 
\end{itemize}
\end{itemize}
\end{frame}

\subsection{Prepare the Observations}
\begin{frame}
\frametitle{Prepare the Observations and Assign the Observational Error}
\begin{equation*}
J(\bf{x})=\frac{1}{2}(\bf{x}-{\color{blue}\bf{x}^b})^T{\color{blue}\bf{B}^{-1}}(\bf{x}-{\color{blue}\bf{x}^b})+\frac{1}{2}({\color{red}\bf{y}}-\textsl{H}(\bf{x}))^T{\color{red}\bf{R}^{-1}}({\color{red}\bf{y}}-\textsl{H}(\bf{x}))
\end{equation*}
\begin{itemize}
\item Conventional observation input for WRFDA is supplied through \pause
\begin{itemize}
\item Little\_R format, observation preprocessor, OBSPROC \pause
\item  Prepbufr format data directly \pause
\end{itemize} \pause
\item Observation error covariance also provided by OBSPROC (R is a diagonal matrix) \pause
\item \alert{Separate input file (ASCII) for radar, both reflectivity and radial velocity} \pause
\item \alert{Separate input file for satellite radiances, BUFR format} 
\end{itemize}
\end{frame}

\subsection{Run WRFDA}
\begin{frame}
\frametitle{Run WRFDA}
\begin{equation*}
J(\bf{x})=\frac{1}{2}(\bf{x}-{\color{blue}\bf{x}^b})^T{\color{blue}\bf{B}^{-1}}(\bf{x}-{\color{blue}\bf{x}^b})+\frac{1}{2}({\color{blue}\bf{y}}-\textsl{H}(\bf{x}))^T{\color{blue}\bf{R}^{-1}}({\color{blue}\bf{y}}-\textsl{H}(\bf{x}))
\end{equation*}
\begin{itemize}
\item $H$ is the observational operator, which calculate the counterpart of observations in model space \pause
\item Conjuagate gradient method
\item Try to find a $\mathbf{x}^a$ , which make the $J$ minimal
\end{itemize}
\end{frame}

\subsection{Run UPDATE\_BC}
\begin{frame}
\frametitle{Update Boundary Condition}
\begin{itemize}
\item After creating an analysis, $\bf{x^a}$, we have changed
the initial conditions for the model \pause
\item However, tendencies in wrfbdy\_d01 file are valid for background, $\bf{x^b}$ \pause
\item The update\_bc program adjusts these tendencies
based on the difference $\bf{x^a - x^b}$ \pause
\item Of course, if $\bf{x^a}$ was produced for reasons other
than running WRF, there is probably not a need to
update boundary conditions
\end{itemize}
\end{frame}

\subsection{FAQ}
\begin{frame}
\frametitle{Frequently Asked Question}
\begin{itemize}
\item Q: What background errors are best for my application? \pause
\item \alert{A: With gen\_be, create your own once you have run your
system for a few weeks to a month, Implement, tune, and iterate}
\end{itemize}
\end{frame}


\section{Question}
\begin{frame}
\frametitle{Question?}
\begin{center}
Questions and Comments ?
\end{center}
\end{frame}
\end{document}
